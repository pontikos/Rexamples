% The document will be printed one side in A4 paper.
\documentclass[12pt,a4paper,twoside]{article}

\usepackage[round]{natbib}

% Packages
% Use Charter font.
%\usepackage{charter}
% for line spacing
\usepackage{setspace}
% Use PDF output.
\usepackage[pdftex]{color,graphicx}
% The output should be wide.
\usepackage{a4wide}
%\usepackage{graphicx}
\usepackage{url}
%for definition list
\usepackage{enumitem}
%for celsius
\usepackage{gensymb}
%for listing code
\usepackage{listings}
%
\usepackage{caption}
\usepackage{subcaption}
%
\usepackage{slashbox}

\usepackage{amsmath}
%puts silly zeroes in section names
%\usepackage{fancyhdr}
%for code snippets
%\usepackage{float}
%\floatstyle{ruled}
%\newfloat{program}{thp}{lop}
%\floatname{program}{Code snippet}

\usepackage{Sweave}

\usepackage{tikz}
\usetikzlibrary{shapes,arrows,calc,through,backgrounds,decorations.pathmorphing,shadows} 

%\usepackage[active,tightpage]{preview}

\usepackage[french, greek, english]{babel}
%input encoding
%\usepackage[iso-8859-7]{inputenc}
%\usepackage[latin1]{inputenc}
%output encoding
\usepackage[T1]{fontenc}
\selectlanguage{english}

% Pour la dedicace
\usepackage{frcursive}

%\usepackage[table]{xcolor}
%\usepackage{booktabs}
%\usepackage{tabu}

\usepackage{url} %tablerules
\usepackage{xr}

\usepackage[labelfont=bf,labelsep=period,justification=raggedright]{caption}
\usepackage{authblk}


%
%\SweaveOpts{keep.source=TRUE,pdf=TRUE,eps=FALSE} 
%\newcommand{\scscst}{\scriptscriptstyle}
%\newcommand{\scst}{\scriptstyle}
%\newcommand{\Rfunction}[1]{{\texttt{#1}}}
%\newcommand{\Rcode}[1]{{\texttt{#1}}}
%\newcommand{\Robject}[1]{{\texttt{#1}}}
%\newcommand{\Rpackage}[1]{{\textsf{#1}}}
%\newcommand{\Rdata}[1]{{\textsf{#1}}}
%\newcommand{\Rclass}[1]{{\textit{#1}}} 
%\usepackage[text={7.5in,9in},centering]{geometry}
%\usepackage{Sweave}
%\setkeys{Gin}{width=0.95\textwidth}
%\usepackage[round]{natbib}
%
\usepackage{graphicx}
\usepackage{url}
\usepackage{hyperref} 
\usepackage{amsmath}


% \usepackage{setspace}
% \setlength{\parindent}{0in}

%\setcounter{secnumdepth}{0}

%\doublespacing

% The document begins here.
\begin{document}
%\SweaveOpts{concordance=TRUE}

\begin{figure}
\begin{center}
\begin{Schunk}
\begin{Soutput}
[1] 704967.8
\end{Soutput}
\end{Schunk}
\includegraphics{clustering-simulation-data}
\end{center}
\caption{ Simulated data of size 2000  }
\label{fig:data}
\end{figure}

\begin{figure}
\begin{center}
\begin{Schunk}
\begin{Soutput}
[1] 532144.4
\end{Soutput}
\end{Schunk}
\includegraphics{clustering-simulation-kmeans}
\end{center}
\caption{ kmeans with $K=5$ }
\label{fig:kmeans}
\end{figure}


\begin{figure}
\begin{center}
\begin{Schunk}
\begin{Soutput}
[1] 503455.4
\end{Soutput}
\end{Schunk}
\includegraphics{clustering-simulation-pam}
\end{center}
\caption{ pam with $K=5$ }
\label{fig:kmeans}
\end{figure}

\begin{figure}
\begin{center}
\begin{Schunk}
\begin{Soutput}
[1] 605137
\end{Soutput}
\end{Schunk}
\includegraphics{clustering-simulation-flowClust}
\end{center}
\caption{ flowClust with $K=5$ }
\label{fig:flowClust}
\end{figure}


\begin{figure}
\begin{center}
\begin{Schunk}
\begin{Soutput}
[1] 579305.9
\end{Soutput}
\end{Schunk}
\includegraphics{clustering-simulation-hclust-ward}
\end{center}
\caption{ hclust ward with $K=5$ }
\label{fig:hclust-ward}
\end{figure}


\begin{figure}
\begin{center}
\begin{Schunk}
\begin{Soutput}
[1] 579305.9
\end{Soutput}
\end{Schunk}
\includegraphics{clustering-simulation-hclust-complete}
\end{center}
\caption{ hclust complete with $K=5$ }
\label{fig:hclust-complete}
\end{figure}




\end{document}

